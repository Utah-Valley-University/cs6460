\DocumentMetadata{testphase=phase-III}
% Toggle solutions: comment out \printanswers to hide solutions and produce a student version.
\documentclass[11pt,addpoints]{exam}

% --- Packages ---
\usepackage[margin=1in]{geometry}
\usepackage{amsmath,amssymb}
\usepackage{graphicx}
\usepackage{booktabs}
\usepackage{enumitem}
\usepackage{tikz}
\usepackage{tcolorbox}

% PDF Accessibility: Enable tagged PDF for ADA compliance
\usepackage{hyperref}
\hypersetup{
    pdftitle={CS 6460 Homework 4},
    pdfauthor={CS 6460},
    colorlinks=false,
    pdfborder={0 0 0},
    pdfstartview={FitH},
    unicode=true
}

% Set document language for accessibility
\usepackage[english]{babel}

% --- Header/Footer ---
\pointsinrightmargin
\bracketedpoints
%% \printanswers % Comment this line to hide solutions

% Fix tagpdf structure issues when solutions are hidden
\ifprintanswers
\else
  \usepackage{etoolbox}
  % Disable automatic paragraph tagging inside solution environments
  \AtBeginEnvironment{solution}{\tagpdfsetup{para/tag=false}}
\fi

% Conditional header based on whether solutions are shown
\ifprintanswers
  \header{CS 6460: Artificial Intelligence}{Key}{Fall 2025}
\else
  \header{CS 6460: Artificial Intelligence}{ }{Fall 2025}
\fi
\footer{}{}{\thepage}

\begin{document}

\begin{center}
  \tagstructbegin{tag=H1}
  \tagmcbegin{tag=H1}
  {\Large \textbf{Homework 4}}\\[6pt]
  \tagmcend
  \tagstructend
\end{center}
\vspace{1em}

\begin{tcolorbox}[colback=gray!10, colframe=gray!50!black, title=\textbf{Instructions}]
  \textit{Points: Please see the points for each problem.}\\
  \textit{Submission: Submit completed homework as a PDF file. Handwritten work or photos of handwritten work must be neat and legible.}
\end{tcolorbox}

\vspace{0.5em}

\begin{tcolorbox}[colback=white, colframe=black, title=\textbf{Points Summary}]
\begin{center}
\begin{tabular}{ccc}
\toprule
\textbf{Question Number} & \textbf{Points Possible} & \textbf{Points Earned} \\
\midrule
1 & 4 & \rule{1.5cm}{0.4pt} \\
2 & 3 & \rule{1.5cm}{0.4pt} \\
3 & 3 & \rule{1.5cm}{0.4pt} \\
\midrule
\textbf{Total} & \textbf{10} & \rule{1.5cm}{0.4pt} \\
\bottomrule
\end{tabular}
\end{center}
\end{tcolorbox}

\vspace{0.5em}

% ============================

Consider a Markov Model with a binary state $X_t \in \{0,1\}$. The transition probabilities are given as follows:

\begin{center}
\begin{tabular}{cc|c}
\toprule
$X_t$ & $X_{t+1}$ & $P(X_{t+1} \mid X_t)$ \\
\midrule
0 & 0 & 0.9 \\
0 & 1 & 0.1 \\
1 & 0 & 0.5 \\
1 & 1 & 0.5 \\
\bottomrule
\end{tabular}
\end{center}

\begin{questions}

  \question[4] The prior belief distribution over the initial state $X_0$ is uniform: $P(X_0 = 1) = P(X_0 = 0) = 0.5$.
  After one timestep, what is the new belief distribution $P(X_1)$?
  
  \begin{solution}
    \textbf{Answer:} $P(X_1=0) = 0.7, P(X_1=1) = 0.3$.
    
    Using the law of total probability:
    \[ P(X_1=x) = \sum_{x_0 \in \{0,1\}} P(X_1=x \mid X_0=x_0) P(X_0=x_0) \]
    Given $P(X_0=0) = 0.5$ and $P(X_0=1) = 0.5$:
    \begin{align*}
    P(X_1=0) &= P(X_1=0 \mid X_0=0)P(X_0=0) + P(X_1=0 \mid X_0=1)P(X_0=1) \\
             &= 0.9(0.5) + 0.5(0.5) \\
             &= 0.45 + 0.25 = 0.7
    \end{align*}
    Since probabilities sum to 1, $P(X_1=1) = 1 - 0.7 = 0.3$.
  \end{solution}

  \uplevel{
    \vspace{1em}
    Now, we incorporate sensor readings. The sensor model is parameterized by a number $\beta \in [0, 1]$:

    \begin{center}
    \begin{tabular}{cc|c}
    \toprule
    $X_t$ & $E_t$ & $P(E_t \mid X_t)$ \\
    \midrule
    0 & 0 & $\beta$ \\
    0 & 1 & $1 - \beta$ \\
    1 & 0 & $1 - \beta$ \\
    1 & 1 & $\beta$ \\
    \bottomrule
    \end{tabular}
    \end{center}
  }

  \question[3] At $t = 1$, we get the first sensor reading, $E_1 = 0$. Use your answer from Question 1 to compute $P(X_1 = 0 \mid E_1 = 0)$. Leave your answer in terms of $\beta$.
  \begin{solution}
    \textbf{Answer:} $\frac{0.7\beta}{0.4\beta + 0.3}$
    
    Using Bayes' Rule:
    \[ P(X_1=0 \mid E_1=0) = \frac{P(E_1=0 \mid X_1=0) P(X_1=0)}{P(E_1=0)} \]
    
    From Question 1, $P(X_1=0) = 0.7$ and $P(X_1=1) = 0.3$.
    From the sensor model:
    \begin{itemize}
        \item $P(E_1=0 \mid X_1=0) = \beta$
        \item $P(E_1=0 \mid X_1=1) = 1 - \beta$
    \end{itemize}
    
    Calculate the denominator (evidence):
    \begin{align*}
    P(E_1=0) &= P(E_1=0 \mid X_1=0)P(X_1=0) + P(E_1=0 \mid X_1=1)P(X_1=1) \\
             &= \beta(0.7) + (1-\beta)(0.3) \\
             &= 0.7\beta + 0.3 - 0.3\beta = 0.4\beta + 0.3
    \end{align*}
    
    Substitute back into Bayes' Rule:
    \[ P(X_1=0 \mid E_1=0) = \frac{0.7\beta}{0.4\beta + 0.3} \]
  \end{solution}

  \question[3] For what range of values of $\beta$ will a sensor reading $E_1 = 0$ increase our belief that $X_1 = 0$? That is, what is the range of $\beta$ for which $P(X_1 = 0 \mid E_1 = 0) > P(X_1 = 0)$?
  \begin{solution}
    \textbf{Answer:} $\beta \in (0.5, 1]$
    
    We want to find $\beta$ such that:
    \[ P(X_1 = 0 \mid E_1 = 0) > P(X_1 = 0) \]
    Using the result from Question 2 and $P(X_1=0)=0.7$:
    \[ \frac{0.7\beta}{0.4\beta + 0.3} > 0.7 \]
    
    Since probabilities and $\beta$ are non-negative, the denominator $0.4\beta + 0.3$ is positive. We can multiply both sides:
    \begin{align*}
    0.7\beta &> 0.7(0.4\beta + 0.3) \\
    \beta &> 0.4\beta + 0.3 \quad \text{(divide by 0.7)} \\
    0.6\beta &> 0.3 \\
    \beta &> 0.5
    \end{align*}
    
    Since $\beta$ is a probability, the range is $\beta \in (0.5, 1]$.
  \end{solution}

\end{questions}

\end{document}
