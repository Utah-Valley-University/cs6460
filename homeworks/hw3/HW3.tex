\DocumentMetadata{testphase=phase-III}
% Toggle solutions: comment out \printanswers to hide solutions and produce a student version.
\documentclass[11pt,addpoints]{exam}

% --- Packages ---
\usepackage[margin=1in]{geometry}
\usepackage{amsmath,amssymb}
\usepackage{graphicx}
\usepackage{booktabs}
\usepackage{enumitem}
\usepackage{tikz}
\usepackage{tcolorbox}

% PDF Accessibility: Enable tagged PDF for ADA compliance
\usepackage{hyperref}
\hypersetup{
    pdftitle={CS 6460 Homework 3},
    pdfauthor={CS 6460},
    colorlinks=false,
    pdfborder={0 0 0},
    pdfstartview={FitH},
    unicode=true
}

% Set document language for accessibility
\usepackage[english]{babel}

% --- Header/Footer ---
\pointsinrightmargin
\bracketedpoints
%% \printanswers % Comment this line to hide solutions

% Fix tagpdf structure issues when solutions are hidden
\ifprintanswers
\else
  \usepackage{etoolbox}
  % Disable automatic paragraph tagging inside solution environments
  \AtBeginEnvironment{solution}{\tagpdfsetup{para/tag=false}}
\fi

% Manually define total points if needed
%\makeatletter
%\renewcommand{\totalpoints}{100}
%\makeatother

% Conditional header based on whether solutions are shown
\ifprintanswers
  \header{CS 6460: Artificial Intelligence}{Key}{Fall 2025}
\else
  \header{CS 6460: Artificial Intelligence}{ - }{Fall 2025}
\fi
\footer{}{}{\thepage}

\begin{document}

\begin{center}
  \tagstructbegin{tag=H1}
  \tagmcbegin{tag=H1}
  {\Large \textbf{Homework 3}}\\[6pt]
  \tagmcend
  \tagstructend
\end{center}
\vspace{1em}

\begin{tcolorbox}[colback=gray!10, colframe=gray!50!black, title=\textbf{Instructions}]
  \textit{Points: Please see the points for each problem.}\\
  \textit{Submission: Submit completed homework as a PDF file. Handwritten work or photos of handwritten work must be neat and legible.}
\end{tcolorbox}

\vspace{0.5em}

\begin{tcolorbox}[colback=white, colframe=black, title=\textbf{Points Summary}]
\begin{center}
\begin{tabular}{ccc}
\toprule
\textbf{Question Number} & \textbf{Points Possible} & \textbf{Points Earned} \\
\midrule
1 & 1 & \rule{1.5cm}{0.4pt} \\
2 & 1 & \rule{1.5cm}{0.4pt} \\
3 & 1 & \rule{1.5cm}{0.4pt} \\
4 & 1 & \rule{1.5cm}{0.4pt} \\
5 & 1 & \rule{1.5cm}{0.4pt} \\
6 & 1 & \rule{1.5cm}{0.4pt} \\
7 & 1 & \rule{1.5cm}{0.4pt} \\
8 & 1 & \rule{1.5cm}{0.4pt} \\
9 & 3 & \rule{1.5cm}{0.4pt} \\
10 & 3 & \rule{1.5cm}{0.4pt} \\
11 & 3 & \rule{1.5cm}{0.4pt} \\
12 & 3 & \rule{1.5cm}{0.4pt} \\
\midrule
\textbf{Total} & \textbf{20} & \rule{1.5cm}{0.4pt} \\
\bottomrule
\end{tabular}
\end{center}
\end{tcolorbox}

\vspace{0.5em}

% ============================
\tagstructbegin{tag=H2}
\tagmcbegin{tag=H2}
\section{Bayes' Nets Representation}
\tagmcend
\tagstructend

\tagstructbegin{tag=H3}
\tagmcbegin{tag=H3}
\textbf{Graph Structure: Conditional Independence}
\tagmcend
\tagstructend

Consider the Bayes' net given below.

\begin{center}
  \begin{figure}[h]
    \centering
    \includegraphics[width=0.5\linewidth]{media/image1.png}
    \caption{Bayes' net graph structure showing nodes A through I and their directed dependencies}
  \end{figure}
\end{center}

Recall:
\begin{itemize}
  \item $X \perp\!\!\!\perp Y$ reads as ``$X$ is independent of $Y$''.
  \item $X \perp\!\!\!\perp Y \mid \{Z, W\}$ reads as ``$X$ is independent of $Y$ given $Z$ and $W$''.
\end{itemize}

For each statement below, indicate whether it is True or False. (1 point each)

\begin{questions}

  \question[1] \textbf{True} \quad \textbf{False}\\[2pt]
  It is guaranteed that $A \perp\!\!\!\perp B$.
  \begin{solution}
    \textbf{Answer:} False
    
    There is a direct edge $A \rightarrow B$ in the Bayes' net. While $B$ depends on $A$ (since $A$ is a parent of $B$), independence is symmetric: $A \perp\!\!\!\perp B$ means both $P(A|B) = P(A)$ and $P(B|A) = P(B)$. Since there is an active path between $A$ and $B$ (the direct edge), they are not independent. Therefore, this statement is \textbf{False}.
  \end{solution}
  
  \question[1] \textbf{True} \quad \textbf{False}\\[2pt]
  It is guaranteed that $A \perp\!\!\!\perp C$.
  \begin{solution}
    \textbf{Answer:} True
    
    The only path between $A$ and $C$ is: $A \rightarrow B \rightarrow E \leftarrow F \leftarrow C$. Node $E$ is a collider on this path. Since $E$ is not observed, the collider blocks the path. With no active paths between $A$ and $C$, they are independent. Therefore, this statement is \textbf{True}.
  \end{solution}
  
  \question[1] \textbf{True} \quad \textbf{False}\\[2pt]
  It is guaranteed that $A \perp\!\!\!\perp D \mid E$.
  \begin{solution}
    \textbf{Answer:} False
    
    There is a path: $A \rightarrow B \rightarrow E \leftarrow D$. Node $E$ is a collider on this path. When we condition on $E$ (observe $E$), the collider becomes activated, making the path active. Therefore, $A$ and $D$ are not independent given $E$. This statement is \textbf{False}.
  \end{solution}
  
  \question[1] \textbf{True} \quad \textbf{False}\\[2pt]
  It is guaranteed that $A \perp\!\!\!\perp I \mid E$.
  \begin{solution}
    \textbf{Answer:} True
    
    The path between $A$ and $I$ is: $A \rightarrow B \rightarrow E \rightarrow I$. Node $E$ is on this chain. When we condition on $E$ (observe $E$), nodes on a chain block the path. Since $E$ blocks the path and there are no other active paths between $A$ and $I$ given $E$, they are independent. Therefore, this statement is \textbf{True}.
  \end{solution}
  
  \question[1] \textbf{True} \quad \textbf{False}\\[2pt]
  It is guaranteed that $B \perp\!\!\!\perp C \mid I$.
  \begin{solution}
    \textbf{Answer:} False
    
    There is a path: $B \rightarrow E \rightarrow I \leftarrow H \leftarrow E \leftarrow F \leftarrow C$. Node $I$ is a collider on this path. When we condition on $I$ (observe $I$), the collider becomes activated, making the path active. Therefore, $B$ and $C$ are not independent given $I$. This statement is \textbf{False}.
    
    Alternatively, we can see: $B \rightarrow E \rightarrow H \rightarrow I$ and $C \rightarrow F \rightarrow E \rightarrow H \rightarrow I$. When $I$ is observed, it activates the collider, creating an active path through $E$ and $H$.
  \end{solution}
  
  \question[1] \textbf{True} \quad \textbf{False}\\[2pt]
  It is guaranteed that $F \perp\!\!\!\perp A \mid H$.
  \begin{solution}
    \textbf{Answer:} True
    
    The path between $F$ and $A$ is: $F \rightarrow E \rightarrow H \leftarrow G \leftarrow D$ (but this doesn't reach $A$). Let's trace more carefully: $F \rightarrow E \leftarrow B \leftarrow A$. On this path, $E$ is a collider. When we condition on $H$ but not on $E$, the collider at $E$ remains blocked. There is no active path between $F$ and $A$ given $H$. Therefore, $F$ and $A$ are independent given $H$. This statement is \textbf{True}.
  \end{solution}
  
  \question[1] \textbf{True} \quad \textbf{False}\\[2pt]
  It is guaranteed that $D \perp\!\!\!\perp I \mid \{E,G\}$.
  \begin{solution}
    \textbf{Answer:} True
    
    The paths between $D$ and $I$ are:
    \begin{itemize}
      \item $D \rightarrow E \rightarrow I$: $E$ is on this chain, so conditioning on $E$ blocks it.
      \item $D \rightarrow E \rightarrow H \rightarrow I$: $E$ is on this chain, so conditioning on $E$ blocks it.
      \item $D \rightarrow G \rightarrow H \rightarrow I$: $G$ is on this chain, so conditioning on $G$ blocks it.
      \item $D \rightarrow E \rightarrow H \leftarrow G$: $H$ is a collider on this path. Since we condition on $G$ but not on $H$, the collider remains blocked.
    \end{itemize}
    All paths between $D$ and $I$ are blocked when conditioning on $\{E,G\}$. Therefore, $D$ and $I$ are independent given $\{E,G\}$. This statement is \textbf{True}.
  \end{solution}
  
  \question[1] \textbf{True} \quad \textbf{False}\\[2pt]
  It is guaranteed that $C \perp\!\!\!\perp H \mid G$.
  \begin{solution}
    \textbf{Answer:} True
    
    The path between $C$ and $H$ is: $C \rightarrow F \rightarrow E \rightarrow H \leftarrow G$. On this path, $H$ is a collider. When we condition on $G$ but not on $H$, the collider at $H$ remains blocked. There is no active path between $C$ and $H$ given $G$. Therefore, $C$ and $H$ are independent given $G$. This statement is \textbf{True}.
  \end{solution}

\end{questions}

% ============================
\newpage
\tagstructbegin{tag=H2}
\tagmcbegin{tag=H2}
\section{Bayes' Net Reasoning}
\tagmcend
\tagstructend

\begin{figure}[ht]
  \centering
  \includegraphics[width=0.95\linewidth]{media/image2.png}
  \caption{Bayes' net for disease testing with disease D and tests A and B, showing conditional probability tables}
\end{figure}

\begin{questions}

  \question[3] What is the probability of having disease D and getting a positive result on test A? \quad $P(+d, +a) =$
  \begin{solution}
    To find the joint probability $P(+d, +a)$, we use the chain rule for Bayes' nets. Since test A depends on disease D in the Bayes' net structure, we can factor this as:
    \[
    P(+d, +a) = P(+d) \times P(+a \mid +d)
    \]
    
    This follows from the chain rule: $P(X, Y) = P(X) \times P(Y \mid X)$ when $X$ is a parent of $Y$ in the Bayes' net.
    
    \textbf{Calculation:}
    \begin{itemize}
      \item $P(+d)$ is the prior probability of having the disease (given in the CPT for node D)
      \item $P(+a \mid +d)$ is the conditional probability of a positive test A result given the disease is present (given in the CPT for node A)
    \end{itemize}
    
    Multiply these two values from the conditional probability tables to get the final answer.
  \end{solution}

  \question[3] What is the probability of not having disease D and getting a positive result on test A? \quad $P(-d, +a) =$
  \begin{solution}
    This represents a false positive case: the test is positive even though the disease is not present. Using the chain rule:
    \[
    P(-d, +a) = P(-d) \times P(+a \mid -d)
    \]
    
    Since $P(-d) = 1 - P(+d)$, we can also write:
    \[
    P(-d, +a) = (1 - P(+d)) \times P(+a \mid -d)
    \]
    
    \textbf{Explanation:}
    \begin{itemize}
      \item $P(-d)$ is the prior probability of not having the disease, which equals $1 - P(+d)$
      \item $P(+a \mid -d)$ is the conditional probability of a positive test A result given the disease is absent (the false positive rate, given in the CPT for node A)
    \end{itemize}
    
    Multiply these values to get the probability of a false positive result.
  \end{solution}

  \question[3] What is the probability of having disease D given a positive result on test A? \quad $P(+d | +a) =$
  \begin{solution}
    This is a conditional probability that requires Bayes' rule. We want to find $P(+d \mid +a)$.
    
    Using Bayes' rule:
    \[
    P(+d \mid +a) = \frac{P(+a \mid +d) \times P(+d)}{P(+a)}
    \]
    
    The denominator $P(+a)$ can be found using the law of total probability:
    \[
    P(+a) = P(+a, +d) + P(+a, -d) = P(+a \mid +d) \times P(+d) + P(+a \mid -d) \times P(-d)
    \]
    
    \textbf{Step-by-step calculation:}
    \begin{enumerate}
      \item Calculate $P(+a, +d) = P(+a \mid +d) \times P(+d)$ (from question 9)
      \item Calculate $P(+a, -d) = P(+a \mid -d) \times P(-d)$ (from question 10)
      \item Calculate $P(+a) = P(+a, +d) + P(+a, -d)$
      \item Apply Bayes' rule: $P(+d \mid +a) = \frac{P(+a, +d)}{P(+a)}$
    \end{enumerate}
    
    This gives the posterior probability of having the disease after observing a positive test result.
  \end{solution}

  \question[3] What is the probability of having disease D given a positive result on test B? \quad $P(+d | +b) =$
  \begin{solution}
    Similar to the previous question, we use Bayes' rule to find $P(+d \mid +b)$:
    \[
    P(+d \mid +b) = \frac{P(+b \mid +d) \times P(+d)}{P(+b)}
    \]
    
    The denominator $P(+b)$ is calculated using the law of total probability:
    \[
    P(+b) = P(+b, +d) + P(+b, -d) = P(+b \mid +d) \times P(+d) + P(+b \mid -d) \times P(-d)
    \]
    
    \textbf{Step-by-step calculation:}
    \begin{enumerate}
      \item Calculate $P(+b, +d) = P(+b \mid +d) \times P(+d)$ using the CPT values
      \item Calculate $P(+b, -d) = P(+b \mid -d) \times P(-d) = P(+b \mid -d) \times (1 - P(+d))$
      \item Calculate $P(+b) = P(+b, +d) + P(+b, -d)$
      \item Apply Bayes' rule: $P(+d \mid +b) = \frac{P(+b, +d)}{P(+b)}$
    \end{enumerate}
    
    This gives the posterior probability of having the disease after observing a positive test B result. Note that this may differ from $P(+d \mid +a)$ if tests A and B have different sensitivity and specificity values.
  \end{solution}

\end{questions}

\end{document}

